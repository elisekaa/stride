%%!TEX root = ./UserManual.tex
\chapter{Concepts and Algorithms}
\label{chap:code}


\section{Introduction}
\label{section:ConceptsIntro}

The model population consists of households, schools, workplaces and
communities, which represent a group of people we define as a ``ContactPool''.
Social contacts can only happen within a ContactPool.
When school or work is off, people stay at home and in their primary
community and can have social contacts with the other members.
During other days, people are present in their household, secondary community
and a possible workplace or school.  



We use a $Simulator$ class to organize the activities from the people in the population.
The ContactPools in a population are grouped into ContactCenters (e.g. the different classes of a school are grouped into one K12School ContactCenter).
These ContactCenters are geographically grouped into a geographical grid (sometimes called GeoGrid)

The $Contact Handler$ performs Bernoulli trials to decide whether a contact
between an infectious and susceptible person leads to disease transmission. 
People transit through Susceptible-Exposed-Infected-Recovered states,
similar to an influenza-like disease.
Each $ContactPool$ contains a link to its members and the $Population$ stores all personal
data, with $Person$ objects.
The implementation is based on the open source model from Grefenstette et al. \cite{grefenstette2013}. 
The household, workplace and school clusters are handled separately from the
community clusters, which are used to model general community contacts. The
$Population$ is a collection of $Person$ objects.


%%%%%%%%%%%%%%%%%%%%%%%%%%%%%%%%%%%%%%%%%%%%%%%%%%%%%%%%%%%%%%%%
% Population
%%%%%%%%%%%%%%%%%%%%%%%%%%%%%%%%%%%%%%%%%%%%%%%%%%%%%%%%%%%%%%%%

\section{Population }
\label{section:gengeopop}

\subsection{Background}
\label{subsection:background}

To explain the algorithms used for generating the geography of the countries and their respective population, we have to introduce some background concepts. 

\begin{description}
    \item[ContactPool]:
        A pool of persons that may contact with each other which in turn may lead to disease transmission.
        We distinguish different a number of types of ContactPools associated with the household, the workplace, 
        the school, \ldots. The Household is a key type because it fixed the home address of a person.
    \item[K-12 student]: 
    	Persons from 3 until 18 years of age that are required (at least in Belgium) to attend school. 
    	Students that skip or repeat years are not accounted for.
    \item[College student]:
        Persons older than 18 and younger than 23 years of age that attend an institution of higher education. 
        For simplicity we group all forms of higher eduction into the same type of ContactCenter, a College. 
        A fraction of college students will attend a college ``close to home'' and the others will attend a 
        college ``far from home''. Most higher educations don't last 6 years, but this way we compensate 
        for changes in the field of study, doctoral studies, advanced masters and repeating a failed year of study.
       \item[Household profile]:
        The composition of households in terms of the number of members and their age is an important 
        factor in the simulation. In this case the profile is not defined by the age of its members or fractions, 
        but through a set of reference households. This set contains a sample of households which is 
        representative of the whole population in their composition.
\end{description}


The population files are created in a separate project and consist of:

\begin{description}
    \item[Households]
    \item[K-12 Schools]
    \item[Colleges]  
    \item[Workplaces]        
    \item[Communities]
        
\end{description}



