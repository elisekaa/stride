%%!TEX root = ./UserManual.tex
\chapter{Software}
\label{chap:software}


%%%%%%%%%%%%%%%%%%%%%%%%%%%%%%%%%%%%%%%%%%%%%%%%%%%%%%%%%%%%%%%%
% System Requirements
%%%%%%%%%%%%%%%%%%%%%%%%%%%%%%%%%%%%%%%%%%%%%%%%%%%%%%%%%%%%%%%%
\section{Source code}
\label{section:source}
The source code is maintained in a GitHub repository \url{https://github.com/lwillem/stride}. We use continuous integration via the TravisCI service. Every new revision is built and tested automatically at commit. Results of this proces can be viewed at \url{https://travis-ci.com/github/lwillem/stride}, where you should look for the master branch. The integration status (of the master branch) is flagged in the GitHub repsitory front page for the project. 

Stride is written in C++ and is portable over Linux and Mac OSX platforms that have a sufficiently recent version of a C++ compiler. To build and install \texttt{Stride}, the following tools need to be available on the system:
\begin{compactitem}
    \item A fairly recent GNU g++ or LLVM clang++
    \item make
    \item A fairly recent CMake
    \item The Boost library
    \item Doxygen  and LaTeX (optional, for documentation only)
\end{compactitem}
A detailed list of current versions of operating system, compiler, build and run tools can be found in the project root directory in the file \texttt{PLATFORMS.txt}.

%%%%%%%%%%%%%%%%%%%%%%%%%%%%%%%%%%%%%%%%%%%%%%%%%%%%%%%%%%%%%%%%
% Installation
%%%%%%%%%%%%%%%%%%%%%%%%%%%%%%%%%%%%%%%%%%%%%%%%%%%%%%%%%%%%%%%
\section{Installation}
\label{section:Installation}

To install the project, first obtain the source code by cloning the code repository to a directory .
The build system for \texttt{Stride} uses the \texttt{CMake} tool. This is used to build and install the software at a high level of abstraction and almost platform independent (see \url{http://www.cmake.org/}).
The project also include a traditional \texttt{make} front to \texttt{CMake}  with 
For those users that do not have a working knowledge of \texttt{CMake}, a front end \texttt{Makefile} has been provided that invokes the appropriate \texttt{CMake} commands. It provide the conventional targets to ``build'', ``install'', ``test'' and ``clean'' the project trough an invocation of make. There is one additional target ``configure'' to set up the \texttt{CMake}/\texttt{make} structure that will actually do all the work.

More details on building the software can be found in the file \texttt{INSTALL.txt} in the project root directory.


%%%%%%%%%%%%%%%%%%%%%%%%%%%%%%%%%%%%%%%%%%%%%%%%%%%%%%%%%%%%%%%%
%
%%%%%%%%%%%%%%%%%%%%%%%%%%%%%%%%%%%%%%%%%%%%%%%%%%%%%%%%%%%%%%%
\section{Documentation}
\label{section:documentation}

The Application Programmer Interface (API) documentation is generated automatically
using the Doxygen tool (see \url{www.doxygen.org}) from documentation instructions
embedded in the code . 
%A copy of this documentation for the latest revision of the code in the GitHub repository can be found online at \url{https://broeckho.github.io/stride/}.

The user manual distributed with the source code has been written in \LaTeX (see \url{www.latex-project.org}).



%%%%%%%%%%%%%%%%%%%%%%%%%%%%%%%%%%%%%%%%%%%%%%%%%%%%%%%%%%%%%%%%
%
%%%%%%%%%%%%%%%%%%%%%%%%%%%%%%%%%%%%%%%%%%%%%%%%%%%%%%%%%%%%%%%
\section{Directory layout}

The project directory structure is very systematic.
Everything used to build the software is stored in the root directory:
\begin{compactitem}
    \item \texttt{main}: Code related files (sources, third party libraries and headers, ...)
      	\begin{itemize}
        		\item \texttt{main/<language>}: source code, per language: cpp, python, R
        		\item \texttt{main/resources}: third party resources included in the project:
        \end{itemize}
    \item \texttt{doc}: documentation files (API, manual, ...)
      	\begin{itemize}
        		\item \texttt{doc/doxygen}: files to generate reference documentation with Doxygen
        		\item \texttt{doc/latex}: files needed to generate the user manual with Latex
        \end{itemize}
    \item \texttt{test}: test related files (scripts, regression files, ...)
\end{compactitem}

%%%%%%%%%%%%%%%%%%%%%%%%%%%%%%%%%%%%%%%%%%%%%%%%%%%%%%%%%%%%%%%%
% File formats
%%%%%%%%%%%%%%%%%%%%%%%%%%%%%%%%%%%%%%%%%%%%%%%%%%%%%%%%%%%%%%%%
\section{File formats}
\label{section:FileFormats}

The Stride software supports different file formats:
\begin{compactdesc}
	\item [CSV] \ \\
	Comma separated values, used for population input data and simulator output.
	\item [JSON] \ \\
	JavaScript Object Notation, an open standard format that uses human-readable text to transmit objects consisting of attribute-value pairs. 	 \mbox{(see \url{www.json.org})}
	\item [TXT] \ \\
	Text files, for the logger.
	\item [XML] \ \\
	Extensible Markup Language, a markup language (both human-readable and machine-readable) that defines a set of rules for encoding documents.
\end{compactdesc}


%%%%%%%%%%%%%%%%%%%%%%%%%%%%%%%%%%%%%%%%%%%%%%%%%%%%%%%%%%%%%%%%
%
%%%%%%%%%%%%%%%%%%%%%%%%%%%%%%%%%%%%%%%%%%%%%%%%%%%%%%%%%%%%%%%%
\section{Testing}
Unit tests and install checks are added to Stride based on Google's ``gtest" framework and CMake's ``ctest" tool.
In addition, the code base contains assertions to verify the simulator logic.
They are activated when the application is built in debug mode and can be used to catch errors at run time.


%%%%%%%%%%%%%%%%%%%%%%%%%%%%%%%%%%%%%%%%%%%%%%%%%%%%%%%%%%%%%%%%
% Results
%%%%%%%%%%%%%%%%%%%%%%%%%%%%%%%%%%%%%%%%%%%%%%%%%%%%%%%%%%%%%%%%
\section{Results}
\label{section:Results}

The software can generates different output files:
\begin{compactdesc}
	\item [cases.csv] \ \\
	Cumulative number of cases per day.
	\item [logfile.txt] \ \\
	Details on transmission and/or social contacts events.
	\item [summary.csv] (optional) \ \\
	Aggregated results on the number of cases, configuration details and timings.
	\item [person.csv] (optional) \ \\
	Individual details on infection characteristics.

\end{compactdesc}


