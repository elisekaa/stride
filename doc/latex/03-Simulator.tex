%%!TEX root = ./UserManual.tex
\chapter{Simulator}
\label{chap:simulator}


%%%%%%%%%%%%%%%%%%%%%%%%%%%%%%%%%%%%%%%%%%%%%%%%%%%%%%%%%%%%%%%%
% Workspace
%%%%%%%%%%%%%%%%%%%%%%%%%%%%%%%%%%%%%%%%%%%%%%%%%%%%%%%%%%%%%%%%
\section{Workspace}

By default, \texttt{Stride} is installed in \texttt{./target/installed/} inside de project directory. This can be modified by setting the \texttt{CMAKE\_INSTALL\_PREFIX} on the CMake command line (see the \texttt{INSTALL.txt} file in the prject root directory) or by using the CMakeLocalConfig.txt file (example file can be found  in \texttt{./src/main/resources/make}).

Compilation and installation of the software creates the following files and directories:
\begin{compactitem}
%
	\item Binaries in directory \texttt{<install\_dir>/bin}
      		\begin{compactitem}
        			\item $stride$: executable.
        			\item $gtester$: regression tests for the sequential code.
        			\item $rStride$: R-based simulation wrapper (under construction)
        	\end{compactitem}
%
    \item Configuration files (xml and json) in directory \texttt{<install\_dir>/config}
      	\begin{compactitem}
			\item $run\_default.xml$: default configuration file for Stride.
        		\item \ldots
        \end{compactitem}
%        
    \item Data files (csv) in directory \texttt{<project\_dir>/data}
      	\begin{compactitem}
 			\item $contact\_matrix\_flanders_conditional_teachers.xml$: Social contact rates for flanders for different locations, conditional upon presence, including teaching activities.
			\item $contact\_matrix\_flanders_conditional_teachers_15min.xml$: Idem as above... but selecting only contact of at least 15 minutes. 
			\item $disease\_xxx$: Disease characteristics (incubation and infectious period) for xxx.
			\item $holidays\_xxx$: Holiday characteristics for xxx.
			\item $pop\_belgium600k\_teachter\_censushh.zip$: compressed folder with population data of 600k people for Belgium, including teaching activities.
			\item $immunity\_xxx$: Age specific immunity profiles.
        \end{compactitem}
%
    \item Documentation files in directory \texttt{./target/installed/doc}
      	\begin{compactitem}
        			\item Reference manual
        			\item User manual
        \end{compactitem}
%
\end{compactitem}

The install directory is also the workspace for \texttt{Stride}. The \texttt{Stride} executable allows you to use a different output directory for each new calculation (see the next section).

%%%%%%%%%%%%%%%%%%%%%%%%%%%%%%%%%%%%%%%%%%%%%%%%%%%%%%%%%%%%%%%%
% Run
%%%%%%%%%%%%%%%%%%%%%%%%%%%%%%%%%%%%%%%%%%%%%%%%%%%%%%%%%%%%%%%%
\newpage
\section{Running the simulator}

From the workspace directory, the simulator can be started  using the command \mbox{``\texttt{./bin/stride}''}. Arguments can be passed to the simulator on the command line:
\begin{verbatim}

USAGE: 
 
   bin/stride  [-c <CONFIGURATION>] [-o <<NAME>=<VALUE>>] ... 
			[-i] [--] [--version] [-h]
               
Where: 
 
   -c <CONFIGURATION>,  --config <CONFIGURATION>
     Specifies the run configuration parameters. The format may be either
     -c file=<file> or -c name=<name>. The first is mostly used and may be
     shortened to -c <file>. The second refers to built-in configurations
     specified by their name.
 
     Defaults to -c file=run_default.xml
 
   -o <<NAME>=<VALUE>>,  --override <<NAME>=<VALUE>>  (accepted multiple
      times)
     Override configuration file parameters with values provided here.
 
   -i,  --installed
     File are read from the appropriate (config, data) directories of the
     stride install directory. If false, files are read and written to the
     local directory. 
 
     Defaults to true.
 
   --,  --ignore_rest
     Ignores the rest of the labeled arguments following this flag.
 
   --version
     Displays version information and exits.
 
   -h,  --help
     Displays usage information and exits.

\end{verbatim}



%%%%%%%%%%%%%%%%%%%%%%%%%%%%%%%%%%%%%%%%%%%%%%%%%%%%%%%%%%%%%%%%
% Wrapper
%%%%%%%%%%%%%%%%%%%%%%%%%%%%%%%%%%%%%%%%%%%%%%%%%%%%%%%%%%%%%%%%
\section{rStride Wrapper}
An R wrapper is under development to perform multiple runs with the C++ executable.

